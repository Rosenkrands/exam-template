\documentclass{beamer}

\setbeamertemplate{section in toc}[sections numbered]
\setbeamertemplate{subsection in toc}[subsections numbered]

% american mathematical society
\usepackage{amsmath,amsthm,amssymb}

% reversed cases enviroment
\newenvironment{rcases}{\left.\begin{aligned}}{\end{aligned}\right\rbrace}

% drawing functionality
\usepackage{tikz}
\usetikzlibrary{matrix,patterns}

% equation numbering
\numberwithin{equation}{section}

% theorem types
\newtheorem{proposition}{Proposition}

% math operators
\DeclareMathOperator*{\argmin}{argmin}
\DeclareMathOperator*{\var}{Var}
\DeclareMathOperator*{\cov}{Cov}
\DeclareMathOperator{\VaR}{VaR}
\DeclareMathOperator{\cvar}{CVaR}
\DeclareMathOperator{\supp}{supp}
\DeclareMathOperator{\dist}{dist}

% additional mathematical fonts
\usepackage{mathrsfs}

% remove navigation bar
\beamertemplatenavigationsymbolsempty

% add slide numbers
\setbeamertemplate{footline}[frame number]

% title
\title{Integration Theory}
\subtitle{Exam}
\author{Kasper Rosenkrands}
\institute{Aalborg University}
\date{S20}

% toc at each section
\AtBeginSection[]
{
  \begin{frame}
    \frametitle{Table of Contents}
    \tableofcontents[currentsection, hideothersubsections]
  \end{frame}
}

% definition of comment function itself
\newcommand{\comment}[1]{
    \begin{center}
        \colorbox{yellow}{
            \textsf{
                \textbf{#1}
            }
        }
    \end{center}
}
\newcommand{\task}[1]{
    \begin{center}
        \colorbox{red}{
            \textsf{
                \textbf{#1}
            }
        }
    \end{center}
}

\begin{document}

\frame{\titlepage}

\begin{frame}
\frametitle{Table of Contents}
\tableofcontents[hideallsubsections]
\end{frame}

\section{Lebesgue integration theory}

\subsection{Definitions}

\begin{frame}\frametitle{{\normalsize \secname} \\ {\large \subsecname}}
    \begin{definition}[$\sigma$-algebra]
        A family $m$ of subsets of a set $X$ is said to be a $\sigma$-algebra if
        \begin{enumerate}
            \item $\emptyset, X$ belongs to $m$
            \item If $A$ belongs to $X$, then the complement $\mathcal{C}(A) = X \setminus A$ belongs to $m$. 
            \item If $(A_n)$ is a sequence of sets in in $X$ then the infinte union belongs to $X$. 
        \end{enumerate}
        An ordered pair $(X, m)$ consisting of a set $X$ and a $\sigma$-algebra $m$ of subsets of $X$ is called measurable space.
    \end{definition}
\end{frame}

\end{document}